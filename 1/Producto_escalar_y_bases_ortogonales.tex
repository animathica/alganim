\documentclass[12pt,dvipsnames]{article}
\setcounter{section}{0}

\usepackage{amsmath,amsthm,amssymb,amsbsy}
\usepackage[spanish,es-tabla]{babel}
\decimalpoint
\usepackage{braket}
\usepackage{color}
\usepackage{enumitem}
\usepackage{fancyhdr}
\usepackage{float}
\usepackage[T1]{fontenc}
\usepackage[margin=1.5cm]{geometry} 
\usepackage{graphicx}
\graphicspath{ {images/} }
\usepackage{hyperref}
\usepackage[utf8]{inputenc}
\usepackage{listings}
\usepackage{lmodern}
\usepackage{multicol}
\usepackage{multirow}
\usepackage{pgfplots}
\usepackage{tabularx}
\usepackage{tcolorbox}
\tcbuselibrary{listings,breakable}
\usepackage{tikz}
\usetikzlibrary{babel}
\usepackage{url}
\usepackage{wrapfig}
\usepackage{xcolor}

\setlength{\parindent}{1em}
\setlength{\parskip}{1em}

\definecolor{NARANJA}{rgb}{1,0.467,0}
\definecolor{VERDE}{rgb}{0.31,1,0}
\definecolor{AZUL}{rgb}{0,0.53,1}
\definecolor{ROJO}{rgb}{1,0,0}

\hypersetup{
    colorlinks=true,
    linkcolor=ROJO,
    filecolor=magenta,      
    urlcolor=AZUL,
}
 
\pgfplotsset{compat=1.15}
 
 \renewcommand{\figurename}{Figura}

\renewcommand{\indexname}{Índice}
\renewcommand{\appendixname}{Apéndice}
\renewcommand{\contentsname}{Contenidos}
\renewcommand{\proofname}{Dem.}
\renewcommand{\tablename}{Tabla.}
\renewcommand\qedsymbol{$\blacksquare$}
\newtheorem{teo}{Teorema}[section]
\newtheorem{cor}{Corolario}[section]
\newtheorem{lem}{Lema}[section]
\newtheorem{defi}{Definición}[section]
\newtheorem{obs}{Observación}[section]
\newtheorem{prop}{Propiedades.}[section]
\newtheorem{ejem}{\textbf{\textit{$\circ \ \text{Ejemplo}$}}}[section]
\newtheorem{axi}{Axioma}[section]

\numberwithin{equation}{section}

%%%%%%%%%%%%%%%%%%%%%%%%%%%%%%%%%%%%%%%%%%%%%%%%%%%%%cajas

\newtcolorbox{post}{colback=white,colframe=red!50!black,
	colbacktitle=red!75!black, title= Postulado.}

\newtcolorbox{enu}{colframe=white!85!black, colback=white, leftrule = 10mm, sharp corners, breakable}

\newtcolorbox{solu}{colframe=black, colback=white, leftrule = 1mm, rightrule = -1mm,toprule = -1mm, bottomrule=-1mm, sharp corners, breakable}

\newtcolorbox{corre}{colframe=red, colback=white, leftrule = 1mm, rightrule = -1mm,toprule = -1mm, bottomrule=-1mm, sharp corners, breakable}

\newtcolorbox{enun}{colframe=gray, colback=white!90!black, leftrule = 1mm, rightrule = 1mm, toprule = -1mm, bottomrule=-1mm, sharp corners, breakable}

%%%%%%%%%%%%%%%%%%%%%%%%%%%%%%%%%%%%%%%%%%%%%%%%%%%%%cajas

%%%%%%%%%%%%%%%%%%%%%%%%%%%%%%%%%%%%%%%%%%%%%%%%%%%%%demarcado de soluciones

%New colors defined below
\definecolor{codegreen}{rgb}{0,0.6,0}
\definecolor{codegray}{rgb}{0.5,0.5,0.5}
\definecolor{codepurple}{rgb}{0.58,0,0.82}
\definecolor{backcolour}{rgb}{0.95,0.95,0.92}

%Code listing style named "mystyle"
\lstdefinestyle{mystyle}{
	backgroundcolor=\color{backcolour},   commentstyle=\color{codegreen},
	keywordstyle=\color{magenta},
	numberstyle=\tiny\color{codegray},
	stringstyle=\color{codepurple},
	basicstyle=\ttfamily\footnotesize,
	breakatwhitespace=false,         
	breaklines=true,                 
	captionpos=b,                    
	keepspaces=true,                 
	numbers=left,                    
	numbersep=5pt,                  
	showspaces=false,                
	showstringspaces=false,
	showtabs=false,                  
	tabsize=2
}

%"mystyle" code listing set
\lstset{style=mystyle}

\newenvironment{sol}{\begin{figure}[H]
		\begin{tikzpicture}
		\filldraw[black] (0,0) circle (3pt);
		\draw[line width = 0.5pt] (0,0) -- (4,0) node[above right]{\textbf{Solución:}};
		\end{tikzpicture}
\end{figure}}{\begin{figure}[H]
		\begin{flushright}
			\begin{tikzpicture}
			\draw[line width = 0.5pt] (0,0)-- (4,0);
			\filldraw (4,0) circle (3pt);
			\end{tikzpicture}
\end{flushright}\end{figure}}

%%%%%%%%%%%% TÍTULO Y OBSERVACIONES %%%%%%%%%%%%
 
\begin{document}

\title{Producto escalar y bases ortogonales \\ (proyecciones y ortogonalidad)}
\date{}
\maketitle
%\tableofcontents

\begin{obs}
    Por definición, cualquier base de un espacio vectorial genera a cada vector del espacio mediante una combinación lineal única. Sin embargo, encontrar los coeficientes necesarios para expresar a un vector arbitrario no nulo como combinación lineal de una base es, en general, un proceso laborioso que en complejidad a medida que aumenta la dimensión del espacio. Definiendo la operación de producto escalar \textemdash y, subsecuentemente, los conceptos de ortogonalidad, conjunto ortogonal y base ortogonal\textemdash \ podemos obtener un resultado poderoso que nos muestra cómo encontrar a dichos coeficientes de forma sencilla, suponiendo que existe una base ortogonal. Este resultado, a su vez, nos permite darle una interpretación geométrica al producto escalar.
\end{obs}

\begin{obs}
Las ideas principales a presentar en este video son:
\begin{enumerate}[label=(\roman*)]
    \item Dada una base de un espacio vectorial de dimensión finita y un vector no nulo arbitrario, el problema de encontrar los coeficientes para expresar a ese vector como combinación lineal de la base no es trivial. En general, puede resolverse computacionalmente mediante sistemas de ecuaciones, pero su complejidad aumenta dependiendo de la dimensión del espacio y de qué tipo de vectores viven en él.
    
    \item El producto escalar es una operación no esencial que, de estar presente, dota a un espacio vectorial de estructura adicional; en particular, su introducción permite definir los conceptos de ortogonalidad y conjunto ortogonal.
    
    \item Si un espacio vectorial con producto escalar tiene dimensón finita y existe una base ortogonal, ésta simplifica enormemente el problema de encontrar los coeficientes necesarios para expresar a un vector arbitrario como combinación lineal de una base.

    \item El resultado anterior nos permite interpretar geométricamente la operación de producto escalar: el producto escalar entre dos vectores es proporcional al número por el cual debemos reescalar a uno de los vectores para obtener la componente del otro vector que ``vive'' en el subespacio vectorial generado por el vector reescalado. En particular, dos vectores son ortogonales si, y sólo si, la proyección vectorial de cualquiera de ellos sobre el otro es nula.
    
    \item La interpretación geométrica anterior es consistente con las propiedades que definen al producto escalar.
\end{enumerate}
\end{obs}

\begin{obs}
Considerando que la serie está enfocada en abordar la descomposición espectral de operadores lineales, creo que \textbf{no} es necesario incluir en ella una discusión sobre métrica ni normas distintas de aquellas inducidas por un producto escalar positivo definido; sin embargo, sería bueno buscar referencias para estos temas y ponerlas en la descripción.
\end{obs}

\begin{obs}
Siendo el primer video de la serie, se dejarán varios ejercicios con la intención de que sean retomados en videos posteriores.
\end{obs}

%%%%%%%%%%%%% PRIMERA ESCENA %%%%%%%%%%%

\newpage
\section{Primera escena}

%%%%%%%%%%%%% SEGUNDA ESCENA %%%%%%%%%%%

\newpage
\section{Segunda escena}

%%%%%%%%%%%%% TERCERA ESCENA %%%%%%%%%%%

\newpage
\section{Tercera escena}

%%%%%%%%%%%%% CUARTA ESCENA %%%%%%%%%%%

\newpage
\section{Cuarta escena}

%%%%%%%%%%%%% QUINTA ESCENA %%%%%%%%%%%

\newpage
\section{Quinta escena}


%%%%%%%%%%%%% ÚLTIMA ESCENA %%%%%%%%%%%%

\newpage
\section{Escena final}

\begin{center}
    Sea $(V,K)$ un espacio vectorial con producto escalar.
\end{center}

Ejercicio 1.1 Demuestra que %Se hace referencia a este ejercicio en el video de Ortogonalización y ortonormalización
\begin{align*}
    \langle \vec{u} + a\vec{w}, \vec{v}\rangle &= \langle \vec{u},\vec{v}\rangle + a\langle \vec{w}, \vec{v}\rangle, \\
    \\
    \langle \vec{u}, \vec{v} + a\vec{w}\rangle &= \langle \vec{u},\vec{v}\rangle + \overline{a}\langle \vec{u}, \vec{w}\rangle
\end{align*} para todo $\vec{u},\vec{v},\vec{w}\in V$, $a\in K$ y dibuja un ejemplo no trivial ($\vec{u},\vec{v},\vec{w}\neq\vec{0}, a\neq 0$) en $\mathbb{R}^2$.

\vspace{5mm}
Ejercicio 1.2 Demuestra que las siguientes condiciones son equivalentes. %Se hace referencia a este ejercicio en el video de Ortogonalización y ortonormalización

\begin{enumerate}[label=(\alph*)]
    \item $\langle \vec{u} , \vec{v} \rangle = 0$ para todo $\vec{u}\in V$.

    \item $\vec{v}=\vec{0}$.

    \item $\langle \vec{v} , \vec{u} \rangle = 0$ para todo $\vec{u}\in V$.
\end{enumerate}

Ejercicio 1.3 Demuestra que todo conjunto ortogonal es linealmente independiente si, y sólo si, no contiene al vector nulo.

\vspace{5mm}
Ejercicio 1.4 

\vspace{5mm}
Pregunta 1.1 ¿Cómo puedes interpretar la propiedad de simetría conjugada del producto escalar en un espacio vectorial complejo?\footnote{Ésta ni yo la sé responder en toda su profundida, pero me parece interesante pensarla.}

\vspace{5mm}
Pregunta 1.2 ¿El resultado de las bases ortogonales será válido en espacios vectoriales con producto escalar de dimensión \emph{infnita}? Argumenta\footnote{Respuesta: no porque, en general, las combinaciones lineales infinitas no están bien definidas; se tendrían que considerar cuestiones de convergencia.}.

\end{document}

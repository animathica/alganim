\documentclass[12pt,dvipsnames]{article}
\setcounter{section}{0}

\usepackage{amsmath,amsthm,amssymb,amsbsy}
\usepackage[spanish,es-tabla]{babel}
\decimalpoint
\usepackage{braket}
\usepackage{color}
\usepackage{enumitem}
\usepackage{fancyhdr}
\usepackage{float}
\usepackage[T1]{fontenc}
\usepackage[margin=1.5cm]{geometry} 
\usepackage{graphicx}
\graphicspath{ {images/} }
\usepackage{hyperref}
\usepackage[utf8]{inputenc}
\usepackage{listings}
\usepackage{lmodern}
\usepackage{multicol}
\usepackage{multirow}
\usepackage{pgfplots}
\usepackage{tabularx}
\usepackage{tcolorbox}
\tcbuselibrary{listings,breakable}
\usepackage{tikz}
\usetikzlibrary{babel}
\usepackage{url}
\usepackage{wrapfig}
\usepackage{xcolor}

\setlength{\parindent}{1em}
\setlength{\parskip}{1em}

\definecolor{NARANJA}{rgb}{1,0.467,0}
\definecolor{VERDE}{rgb}{0.31,1,0}
\definecolor{AZUL}{rgb}{0,0.53,1}
\definecolor{ROJO}{rgb}{1,0,0}

\hypersetup{
    colorlinks=true,
    linkcolor=ROJO,
    filecolor=magenta,      
    urlcolor=AZUL,
}
 
\pgfplotsset{compat=1.15}
 
 \renewcommand{\figurename}{Figura}

\renewcommand{\indexname}{Índice}
\renewcommand{\appendixname}{Apéndice}
\renewcommand{\contentsname}{Contenidos}
\renewcommand{\proofname}{Dem.}
\renewcommand{\tablename}{Tabla.}
\renewcommand\qedsymbol{$\blacksquare$}
\newtheorem{teo}{Teorema}[section]
\newtheorem{cor}{Corolario}[section]
\newtheorem{lem}{Lema}[section]
\newtheorem{defi}{Definición}[section]
\newtheorem{obs}{Observación}[section]
\newtheorem{prop}{Propiedades.}[section]
\newtheorem{ejem}{\textbf{\textit{$\circ \ \text{Ejemplo}$}}}[section]
\newtheorem{axi}{Axioma}[section]

\numberwithin{equation}{section}

%%%%%%%%%%%%%%%%%%%%%%%%%%%%%%%%%%%%%%%%%%%%%%%%%%%%%cajas

\newtcolorbox{post}{colback=white,colframe=red!50!black,
	colbacktitle=red!75!black, title= Postulado.}

\newtcolorbox{enu}{colframe=white!85!black, colback=white, leftrule = 10mm, sharp corners, breakable}

\newtcolorbox{solu}{colframe=black, colback=white, leftrule = 1mm, rightrule = -1mm,toprule = -1mm, bottomrule=-1mm, sharp corners, breakable}

\newtcolorbox{corre}{colframe=red, colback=white, leftrule = 1mm, rightrule = -1mm,toprule = -1mm, bottomrule=-1mm, sharp corners, breakable}

\newtcolorbox{enun}{colframe=gray, colback=white!90!black, leftrule = 1mm, rightrule = 1mm, toprule = -1mm, bottomrule=-1mm, sharp corners, breakable}

%%%%%%%%%%%%%%%%%%%%%%%%%%%%%%%%%%%%%%%%%%%%%%%%%%%%%cajas

%%%%%%%%%%%%%%%%%%%%%%%%%%%%%%%%%%%%%%%%%%%%%%%%%%%%%demarcado de soluciones

%New colors defined below
\definecolor{codegreen}{rgb}{0,0.6,0}
\definecolor{codegray}{rgb}{0.5,0.5,0.5}
\definecolor{codepurple}{rgb}{0.58,0,0.82}
\definecolor{backcolour}{rgb}{0.95,0.95,0.92}

%Code listing style named "mystyle"
\lstdefinestyle{mystyle}{
	backgroundcolor=\color{backcolour},   commentstyle=\color{codegreen},
	keywordstyle=\color{magenta},
	numberstyle=\tiny\color{codegray},
	stringstyle=\color{codepurple},
	basicstyle=\ttfamily\footnotesize,
	breakatwhitespace=false,         
	breaklines=true,                 
	captionpos=b,                    
	keepspaces=true,                 
	numbers=left,                    
	numbersep=5pt,                  
	showspaces=false,                
	showstringspaces=false,
	showtabs=false,                  
	tabsize=2
}

%"mystyle" code listing set
\lstset{style=mystyle}

\newenvironment{sol}{\begin{figure}[H]
		\begin{tikzpicture}
		\filldraw[black] (0,0) circle (3pt);
		\draw[line width = 0.5pt] (0,0) -- (4,0) node[above right]{\textbf{Solución:}};
		\end{tikzpicture}
\end{figure}}{\begin{figure}[H]
		\begin{flushright}
			\begin{tikzpicture}
			\draw[line width = 0.5pt] (0,0)-- (4,0);
			\filldraw (4,0) circle (3pt);
			\end{tikzpicture}
\end{flushright}\end{figure}}

%%%%%%%%%%%% TÍTULO Y OBSERVACIONES %%%%%%%%%%%%
 
\begin{document}

\title{Producto escalar y bases ortogonales (proyecciones y ortogonalidad)}
\date{}
\maketitle
%\tableofcontents

\begin{obs}

Las ideas principales a presentar en este video son:
\begin{enumerate}[label=(\roman*)]
    
    \item el producto escalar es una operación no esencial que, de estar presente, dota a un espacio vectorial de estructura adicional;
    
    \item por definición, el producto escalar positivo definido induce nociones de longitud, proyecciones y ortogonalidad;
    
    \item la norma nos da una noción de longitud \textemdash o magnitud, en contextos físicos\textemdash \ y nos permite hacer comparaciones entre vectores;
    
    \item los productos escalares positivo definidos inducen un caso particular de norma;
    
    \item todo vector no nulo puede ser reescalado de tal forma que su norma sea unitaria.
\end{enumerate}

El objetivo central es mostrar cómo los espacios vectoriales con productos escalares positivo definidos, con ayuda de la norma inducida, nos permiten proyectar vectores de manera perpendicular para obtener la componente de uno de ellos a lo largo del eje del otro.
\end{obs}

\begin{obs}
Considerando que la serie está enfocada en abordar la descomposición espectral de operadores lineales, creo que \textbf{no} es necesario incluir en ella una discusión sobre métrica ni normas distintas de aquellas inducidas por un producto escalar positivo definido; sin embargo, sería bueno buscar referencias para estos temas y ponerlas en la descripción.
\end{obs}

\begin{obs}
Siendo el primer video de la serie, se dejarán varios ejercicios con el objetivo de que el primero esté muy ligado con la exposición y cada uno de los siguientes se retome en un video posterior.
\end{obs}

\begin{obs}
Hay que alinear las cosas mejor de lo que logro hacerlo con el comando align*...
\end{obs}

Escenas:

\begin{enumerate}
    \item Introducción (operaciones esenciales y no esenciales) 
    \item Propiedades del producto escalar
    \item Norma inducida, magnitud y normalización de vectores
\end{enumerate}

%%%%%%%%%%%%% PRIMERA ESCENA %%%%%%%%%%%

\newpage
\section{Primera escena}


%%%%%%%%%%%%% ÚLTIMA ESCENA %%%%%%%%%%%%

\newpage
\section{Escena final}

\begin{center}
    Sea $(V,K)$ un espacio vectorial con producto escalar.
\end{center}

Ejercicio 1.1: Demuestra que
\begin{align*}
    \langle \vec{u} + a\vec{w}, \vec{v}\rangle &= \langle \vec{u},\vec{v}\rangle + a\langle \vec{w}, \vec{v}\rangle, \\
    \\
    \langle \vec{u}, \vec{v} + a\vec{w}\rangle &= \langle \vec{u},\vec{v}\rangle + \overline{a}\langle \vec{u}, \vec{w}\rangle
\end{align*} para $\vec{u},\vec{v},\vec{w}\in V$, $a\in K$ y dibuja un ejemplo no trivial ($\vec{u},\vec{v},\vec{w}\neq\vec{0}, a\neq 0$) en $\mathbb{R}^2$.

\hspace{5mm}

Ejercicio 1.2: Demuestra que
\begin{align*}
    \bigg\langle \vec{u}, \vec{v} - \frac{\langle \vec{u}, \vec{v}\rangle}{||\vec{u}||}\hat{u} \bigg\rangle = 0
\end{align*} para $\vec{u},\vec{v}\in V$ con $\vec{u},\vec{v}\neq\vec{0}$ y dibuja un ejemplo en $\mathbb{R}^2$.

\vspace{5mm}

Ejercicio 1.3: 

\vspace{5mm}

Ejercicio 1.4: Demuestra que
\begin{align*}
    \langle \vec{u}, \vec{v} \rangle = 0 \quad \forall \  \vec{v}\in V \implies \vec{u}=\vec{0}, \\
    \\
    \langle \vec{u}, \vec{v} \rangle = 0 \quad \forall \  \vec{u}\in V \implies \vec{v}=\vec{0}. \\
\end{align*}

\vspace{5mm}

Ejercicio 1.5:

\vspace{5mm}

Ejercicio 1.6:

\vspace{5mm}

Pregunta:

\end{document}

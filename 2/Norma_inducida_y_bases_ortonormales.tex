\documentclass[12pt,dvipsnames]{article}
\setcounter{section}{0}

\usepackage{amsmath,amsthm,amssymb,amsbsy}
\usepackage[spanish,es-tabla]{babel}
\decimalpoint
\usepackage{braket}
\usepackage{color}
\usepackage{enumitem}
\usepackage{fancyhdr}
\usepackage{float}
\usepackage[T1]{fontenc}
\usepackage[margin=1.5cm]{geometry} 
\usepackage{graphicx}
\graphicspath{ {images/} }
\usepackage{hyperref}
\usepackage[utf8]{inputenc}
\usepackage{listings}
\usepackage{lmodern}
\usepackage{multicol}
\usepackage{multirow}
\usepackage{pgfplots}
\usepackage{soul}
\usepackage{tabularx}
\usepackage{tcolorbox}
\tcbuselibrary{listings,breakable}
\usepackage{tikz}
\usetikzlibrary{babel}
\usepackage{url}
\usepackage{wrapfig}
\usepackage{xcolor}

\setlength{\parindent}{1em}
\setlength{\parskip}{1em}

\definecolor{NARANJA}{rgb}{1,0.467,0}
\definecolor{VERDE}{rgb}{0.31,1,0}
\definecolor{AZUL}{rgb}{0,0.53,1}
\definecolor{ROJO}{rgb}{1,0,0}

\hypersetup{
    colorlinks=true,
    linkcolor=ROJO,
    filecolor=magenta,      
    urlcolor=AZUL,
}
 
\pgfplotsset{compat=1.15}
 
\renewcommand{\figurename}{Figura}

\newcommand{\anim}[2]{\textcolor{red}{\textbf{\hl{#1}}}\footnote{#2}}

\renewcommand{\indexname}{Índice}
\renewcommand{\appendixname}{Apéndice}
\renewcommand{\contentsname}{Contenidos}
\renewcommand{\proofname}{Dem.}
\renewcommand{\tablename}{Tabla.}
\renewcommand\qedsymbol{$\blacksquare$}
\newtheorem{teo}{Teorema}[section]
\newtheorem{cor}{Corolario}[section]
\newtheorem{lem}{Lema}[section]
\newtheorem{defi}{Definición}[section]
\newtheorem{obs}{Observación}[section]
\newtheorem{prop}{Propiedades.}[section]
\newtheorem{ejem}{\textbf{\textit{$\circ \ \text{Ejemplo}$}}}[section]
\newtheorem{axi}{Axioma}[section]

\numberwithin{equation}{section}

%%%%%%%%%%%%%%%%%%%%%%%%%%%%%%%%%%%%%%%%%%%%%%%%%%%%%cajas

\newtcolorbox{post}{colback=white,colframe=red!50!black,
	colbacktitle=red!75!black, title= Postulado.}

\newtcolorbox{enu}{colframe=white!85!black, colback=white, leftrule = 10mm, sharp corners, breakable}

\newtcolorbox{solu}{colframe=black, colback=white, leftrule = 1mm, rightrule = -1mm,toprule = -1mm, bottomrule=-1mm, sharp corners, breakable}

\newtcolorbox{corre}{colframe=red, colback=white, leftrule = 1mm, rightrule = -1mm,toprule = -1mm, bottomrule=-1mm, sharp corners, breakable}

\newtcolorbox{enun}{colframe=gray, colback=white!90!black, leftrule = 1mm, rightrule = 1mm, toprule = -1mm, bottomrule=-1mm, sharp corners, breakable}

%%%%%%%%%%%%%%%%%%%%%%%%%%%%%%%%%%%%%%%%%%%%%%%%%%%%%cajas

%%%%%%%%%%%%%%%%%%%%%%%%%%%%%%%%%%%%%%%%%%%%%%%%%%%%%demarcado de soluciones

%New colors defined below
\definecolor{codegreen}{rgb}{0,0.6,0}
\definecolor{codegray}{rgb}{0.5,0.5,0.5}
\definecolor{codepurple}{rgb}{0.58,0,0.82}
\definecolor{backcolour}{rgb}{0.95,0.95,0.92}

%Code listing style named "mystyle"
\lstdefinestyle{mystyle}{
	backgroundcolor=\color{backcolour},   commentstyle=\color{codegreen},
	keywordstyle=\color{magenta},
	numberstyle=\tiny\color{codegray},
	stringstyle=\color{codepurple},
	basicstyle=\ttfamily\footnotesize,
	breakatwhitespace=false,         
	breaklines=true,                 
	captionpos=b,                    
	keepspaces=true,                 
	numbers=left,                    
	numbersep=5pt,                  
	showspaces=false,                
	showstringspaces=false,
	showtabs=false,                  
	tabsize=2
}

%"mystyle" code listing set
\lstset{style=mystyle}

\newenvironment{sol}{\begin{figure}[H]
		\begin{tikzpicture}
		\filldraw[black] (0,0) circle (3pt);
		\draw[line width = 0.5pt] (0,0) -- (4,0) node[above right]{\textbf{Solución:}};
		\end{tikzpicture}
\end{figure}}{\begin{figure}[H]
		\begin{flushright}
			\begin{tikzpicture}
			\draw[line width = 0.5pt] (0,0)-- (4,0);
			\filldraw (4,0) circle (3pt);
			\end{tikzpicture}
\end{flushright}\end{figure}}

%%%%%%%%%%%%%%%%%%%%%%%%%%%%%%%%%%%%%%%%%%%%%%%%%%%%%demarcado de soluciones
 
\begin{document}

\title{Norma inducida y bases ortonormales}
\date{}
\maketitle
%\tableofcontents

\begin{obs}
    Las ideas principales a presentar en este video son:

    \begin{enumerate}[label=(\roman*)]
        \item Por definición, un producto escalar positivo en un espacio vectorial induce una norma en ese espacio. Aunque no todas las normas provienen de un producto escalar positivo definido\footnote{Colocar ejemplos de normas no inducidas por productos escalares positivo definidos en la descripción.}, en esta serie de videos nos enfocaremos en este tipo de normas.

        \item La norma nos da una noción de \emph{magnitud} de un vector, y nos permite hacer comparaciones entre vectores en este sentido.

        \item Todo vector no nulo puede ser reescalado de tal forma que el vector resultante tenga norma unitaria. A este tipo de vectores se les conoce como vectores unitarios. En particular, un conjunto en el que todos los vectores son unitarios y ortogonales entre sí se conoce como conjunto ortonormal.

        \item Si un espacio vectorial con producto escalar tiene dimensión finita y existe una base ortonormal, entonces el problema de encontrar los coeficientes necesarios para expresar a un vector arbitrario como combinación lineal de esta base tiene una solución extremadamente simple.
    \end{enumerate}
\end{obs}

%%%%%%%%%%%%%%%%%%%%%%%%%% PRIMERA ESCENA %%%%%%%%%%%%%%%%%%%%%%%%%%%%%%%%%

\newpage
\section{Primera escena}

%%%%%%%%%%%%%%%%%%%%%%%%%% SEGUNDA ESCENA %%%%%%%%%%%%%%%%%%%%%%%%%%%%%%%%%

\newpage
\section{Segunda escena}

%%%%%%%%%%%%%%%%%%%%%%%%%% TERCERA ESCENA %%%%%%%%%%%%%%%%%%%%%%%%%%%%%%%%%

\newpage
\section{Tercera escena}

Hablar sobre la notación $\frac{1}{||\vec{u}||}\vec{u}\equiv \frac{\vec{u}}{||\vec{u}||}\equiv \hat{u}$ para $\vec{u}\neq\vec{0}$ y demostrar que $||\hat{u}|| = 1$.

%%%%%%%%%%%%%%%%%%%%%%%%%% CUARTA ESCENA %%%%%%%%%%%%%%%%%%%%%%%%%%%%%%%%%

\newpage
\section{Cuarta escena}

%%%%%%%%%%%%%%%%%%%%%%%%%% ÚLTIMA ESCENA %%%%%%%%%%%%%%%%%%%%%%%%%%%%%%%%%

\newpage
\section{Escena final}

%\begin{center}
%    Sea $V$ un espacio vectorial con producto escalar.
%\end{center}

Ejercicio 2.1 Demuestra que todo conjunto ortonormal es linealmente independiente. \\

Ejercicio 2.2 \\

Ejercicio 2.¿3? Una \emph{métrica} o \emph{función de distancia} en un conjunto $X$ es una función $f(\cdot, \cdot ):X\times X\to [0,\infty)$ que cumple las siguientes propiedades:
\begin{enumerate}[label=(\roman*)]
    \item $d(x,y)=0$ si, y sólo si, $x=y$;

    \item $d(x,y)=d(y,x)$ para todo $x,y\in X$;

    \item $d(x,y) \le d(x,z) + d(z,y)$ para todo $x,y,z\in X$.
\end{enumerate}

\noindent Demuestra que si $(V,K)$ es un espacio vectorial con norma $||\cdot||$, entonces la función dada por $d(\vec{x},\vec{y})=||\vec{x}-\vec{y}||$ para todo $\vec{x},\vec{y}\in V$ es una métrica en $V$. Por lo tanto, toda norma \emph{induce} una métrica. En particular, se sigue que todo producto escalar positivo definido induce una métrica; sin embargo, existen métricas que no son inducidas por normas ni productos escalares. \\

Pregunta 2.¿4?

\end{document}
